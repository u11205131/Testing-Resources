\documentclass[a4paper]{article}

\title{Project Implementation
\\COS 301 Buzz Project
\\Group: Testing Phase: Resources
\\Version 1.0}

\author{
\\
\\Carla de Beer 95151835
\\Prenolan Govender 13102380
\\Shaun Meintjies 13310896
\\Collins Mphahlele 12211070
\\Dumisani Msiza 12225887
\\Joseph Murray 12030733
\\Sifiso Shabangu 12081622
\\Joseph Potgieter 12003672
\\Johan van Rooyen 11205131
\\
\\
\textit{https://github.com/Carla-de-Beer/Testing-Resources}
\\
\\ University of Pretoria}

\date{24 April 2015}

% Need to compile using XeLATeX to avoid default font
\usepackage{fontspec}
\setmainfont{Arial}
% Graphics settings for .eps files
\usepackage{graphicx}
\usepackage{epstopdf}
\DeclareGraphicsExtensions{.eps}
\usepackage{float}

\begin{document}

\maketitle
% No page number to cover page
\pagenumbering{gobble}
\newpage
% start page numbering
\pagenumbering{arabic}

% Generate Table of Contents

\tableofcontents
\newpage

\section{Introduction}

The purpose of this task was to test the functionality provided by the Resources teams for the Buzz Project.
\\
\\
According to the master specification document, version 0.1, released 13 March 2015, the buzzResources module is used to upload and manage resources like media (e.g. images, video) and documents (e.g. PDF documents, Open Document Format documents). These resources are to be either embedded or linked to in posts.
\\
\\
This team took each of the pre and post-conditions of the required use case, considering the work of both Resources A and Resources B, and tested them for compliance with the functional requirements. 
\\
\\The use case for this team is called \textbf{uploadResource}. In terms of the requirements for this use case, and as defined by the master specification document, users should be able to upload resources such as media files or documents. Any uploaded resource should be accessible by other users who should also be able to specify links to that resource. 
\\
\\The functional requirements preconditions for this use case included the need to
\begin{itemize}
\item detect the mime type
\item check that size constraints are met
\item check that the resource type is supported
\end{itemize}
The functional requirements postconditions for this use case included the need to
\begin{itemize}
\item check that the resource persisted
\item check that the URL for the resource was created
\end{itemize}

\section {Functional Testing}

\subsection {Resources A}
\subsubsection {Pre-condition violations}
\subsubsection {Post-condition violations}
\subsubsection {Data structure requirements}

\subsection {Resources B}
\subsubsection {Pre-condition violations}

\paragraph{removeResource}
The pre-condition of removeResource is assumed to refer to the identification number of a resource in the database which should exist prior to this execution. There are no violations of this pre-condition due to error checks being performed to verify whether or not the entry exists.

\subsubsection {Post-condition violations}

\paragraph{removeResource}
The post-condition of removeResource would be the removal of an entry. Due to a lack of unit testing functionality, the post-condition fails as there is no visible indication of removeResource working.

\subsubsection {Data structure requirements}

\paragraph{removeResource}
No data structure requirements mentioned in the service contract, nor does removeResource utilise any data structures.

\section {Non-functional testing / assessment}
\subsection {Resources A}
The use case was tested against the following list of architectural requirements:
\subsubsection {Usability}
\subsubsection {Performance}
\subsubsection {Scalability}
\subsubsection {Testability}
\subsubsection {Security}
\subsubsection {Reliability}
\subsubsection {Reusability}
\subsubsection {Pluggability}


\subsection {Resources B}
The use case was tested against the following list of architectural requirements:
\subsubsection {Usability}

\paragraph{removeResource}
removeResource demonstrates usability by being simple and easy to use. The function only needs be called along with the ID of the resource to be removed which is intuitive.

\subsubsection {Performance}

\paragraph{removeResource}
removeResource does not suffer from any performance issues in and of itself, however should the database maintain a sizeable stature, removeResource will be seen to suffer in performance. This is to be expected.

\subsubsection {Scalability}

\paragraph{removeResource}
removeResource offers no solutions to accomodate an increase or decrease in size of the database.

\subsubsection {Testability}

\paragraph{removeResource}
removeResource has a complete lack of testability due to it not conforming to proper unit testing standards (making use of callbacks, depending on a live database etc.).

\subsubsection {Security}

\paragraph{removeResource}
The only security concern would be an incorrect ID being used in order to remove a different resource than required, potentially putting the system at risk.

\subsubsection {Reliability}

\paragraph{removeResource}
According to the failure of the post-condition, removeResource gives a poor indication with regards to reliability.

\subsubsection {Reusability}

\paragraph{removeResource}
removeResource is not very reusable due to the dependence on a single database. Using this function throughout this system would not be possible/practical.

\subsubsection {Pluggability}

\paragraph{removeResource}
In the same vein as reusability, removeResource cannot be integrated into other systems due to a lack of robustness and versatility.


\section {Critical Evaluation and Recommendations}
\subsection {Resources A}
\subsection {Resources B}

% \paragraph{removeResource}
%removeResource is logically correct, however the testability of the function is an important aspect which seems to have been overlooked. In addition, the function is not versatile and cannot be adapted easily into another system. It is recommended that removeResource be looked at again and perhaps redone to conform to acceptable testing practices and allow for more robust and functional code. 


\end{document}
